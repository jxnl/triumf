\documentclass[english, 11pt]{article}
\usepackage{notes}

\newcommand{\thiscoursecode}{TRIUMF SEMINAR NOTES}
\newcommand{\thiscoursename}{Student Seminars}
\newcommand{\thisprof}{Staff}
\newcommand{\me}{Jason Liu}
\newcommand{\thisterm}{(Winter) 2014}
\newcommand{\website}{lithium11.com}

% Headers
\chead{\thiscoursename}
\lhead{\thisterm}

% Title
\newcommand{\notefront} {
\pagenumbering{roman}
\begin{center}

{\ttfamily \url{\website}} {\small}
\textbf{\Huge{\noun{\thiscoursecode}}}{\Huge \par}
{\large{\noun{\thiscoursename}}}\\ \vspace{0.1in}
{\noun \thisprof} \ $\bullet$ \ {\noun \thisterm} \ $\bullet$ \ {\noun {Jason Liu}} \\
\end{center}
}

% Begin Document
\begin{document}

% Notes front
\notefront
% Table of Contents and List of Figures
\tocandfigures
% Abstract
\doabstract{These notes are intended as a resource for myself; past, present, or future students of this course, and anyone interested in the material. The goal is to provide an end-to-end resource that covers all material discussed in the course displayed in an organized manner. If you spot any errors or would like to contribute, please contact me directly.}


% Vijay K. Verma
\section{Project Managment}

The goal of this workshop is to do a brief overview of PMI, project management and process groups. Balancing project requirements and influence of the uncertainty.

\subsection*{What is Unique in Scientific Organizations}

Unlike some pure engineering paths, there is more uncertainty in research. We have limited funding with bleeding edge technology. Moreover, we have specialized people and skill mixes along with special constraints like radiation, dose limits etc.. Special challenges also occur like regulations, international collaborations and political ones.

\subsection*{Learning Objectives}

\begin{itemize}
  \item What is \nameref{project management}
  \item PMBOK knowledge areas and constraints
  \item Project management processes
  \item \nameref{plc}
  \item \nameref{pmbok}
\end{itemize}

\begin{defn}[Project]\label{project}
A temporary endeavor undertaken to unique project or service.
\end{defn}

\begin{defn}[Project Management]\label{project management}

Application of knowledge, skills, tools and techniques to project activities in order to meet or exceed stakeholder needs and expectations from a \nameref{project}. We need to understand what people want without them knowing. They will want it, so deliver.
\end{defn}

\subsection{Project Manager}
slides(ex)

\subsection{Processes of PMI}

\begin{enumerate}
  \item Initiating: Authorizing the project. Spending resources to start.
  \item Planning: Defining and refining objectives, select best course of action.
  \item Executing: Coordinating and integrating people to carry out plan.
  \item Monitoring and Control: Measure progress, identify variance and cause to take corrective actions.
  \item Closing: Formalize acceptance of project and bring to orderly end.
\end{enumerate}

\subsubsection*{Crazies}

People and money and others all effect the project. We also need strong a strong champion. Things like priorities and stakeholders keep changing. We need to figure out the network and get things done in the end. Management only hears what they want to hear. They can't hire or fire any thing. You need to give conditions that lead to success. You can't say no.

\subsubsection{PMBOK Definitions}\label{pmbok}

\begin{defn}[Project Scope]\label{scope}
The work that needs to be accomplished to deliver a product, service, or result with the specified features and functions.
\end{defn}

\begin{defn}[Project Life Cycle]\label{plc}
Collection of sequential project phases who name and number are determined by the control needs of the organizations involved in the project. It changes by the industry.
\end{defn}

\begin{defn}[Work Breakdown Structure]\label{wbs}
A deliverable-oriented grouping of the project elements which organizes and defines the total scope of the project.It helps organize the total scope of the project. Divide work into stand-alone. Often in the form of a tree diagram.
\end{defn}

\begin{defn}[Organization Breakdown Structure]\label{obs}
Project organization relating work packages to organizational units or people responsible. On top of the tree diagram of \nameref{wbs}, each block refers to the person responsible.
\end{defn}

\begin{defn}[Project Phase]\label{project phase}
A collection of related \nameref{activity}, usually culminating in the completion of a major deliverable
\end{defn}

\begin{defn}[Activity]\label{activity}
An element of work performed during the course of a \nameref{project phase} usually has a duration and expected cost and expected resources requirements, and be subdivided into tasks.
\end{defn}

\begin{defn}[Task]\label{task}
The lowest level of effort on a project, subdivided \nameref{activity}. Ones all tasks are finished, one has achieved a milestone.
\end{defn}

\begin{defn}[Milestone]\label{milestone}
A significant event in the \nameref{project} usually completion of a major.
\end{defn}

\subsubsection{Defining Success}

\begin{itemize}
  \item Accepted by stakeholders and it just either meet or exceed the expectations
  \item Meeting the performance requirement (scope)
  \item On time and in budget
  \item Minimum change in scope and does not corrupt workflow
\end{itemize}

\subsection{Time Management}

There are some key processes required to ensure timely completion of the projects. We have to {\bf define} the \nameref{activity} that must be performed. We must {\bf sequence} events and identify and document dependencies. We also must {\bf estimate} the number of work periods which will be need to complete individual activities. Schedule development and control is also important.

\subsubsection{Planning Approach?}

\begin{enumerate}
    \item Define \nameref{scope}
    \item Develop \nameref{wbs}
    \item Develop \nameref{obs}
    \item Prepare Responsibility Matrix
    \item Prepare Project Cost Estimate
    \item Define Quality Requirements
    \item Identify activity sequence and dependencies
    \item Identify the critical path and calculate project duration
    \item Analyze risk
    \item Assemble Project Plan and get Approval
\end{enumerate}

\subsubsection{Planning Benefits?}

The benefits of planning this way is that it is an interactive approach to ensure nothing is overlooked. It allows for better communication between stakeholders and it serves as the starting point for project approval. With planning done right, changes in scope and be noted and risks are addressed. By documenting this process, it becomes easier to justify the requirements of the project. \\

\subsubsection{Statement of Work}
The following is the contents of what a general statement of work.
\begin{itemize}
    \item Purpose of Project
    \item Scope Statement
    \item Product Definition (deliverables)
    \item Cost \& Schedule Estimates
    \item Project Organization Structure
    \item \nameref{ram}
    \item Risks
    \item Assumptions
    \item Project Success Factors
\end{itemize}

\subsubsection{Responsibility Matrix }\label{ram}

\begin{exmp}[Software Development]
This is a Responsibility Assignment Matrix. B - Responsible, A = Accountable, C = Consult, I = Inform.
\begin{center}
\begin{tabular}{|l|l|l|l|l|}
  \hline
  ~                  & Kelly & Jason & Jonah & Jake \\ \hline
  Define Requirements & C     & R     & A     & A    \\
  Functional Specs.  & I     & A     & C     & R    \\
  Coding             & I     & C     & R     & A    \\
  Documentation      & C     & R     & A     & I    \\
  Systems Design     & R     & A     & C     & I    \\ \hline
\end{tabular}
\end{center}
\end{exmp}

\subsubsection{Milestones}

It is often important to define milestones in any project. A time to celebrate completion of a portion of the project. With this in mind, we need to figure out different ways to tracking a project. Bar charts are one way of figure out what needs to be done. They are easy to prepare and update but they can become complex and does not help with the control of the project. One better method to use would be the \nameref{cpm}.

\subsubsection{Critical Path Method} \label{cpm}

\begin{defn}[Critical Path] \label{cp}
  The longest path through the network. It has zero float defined by a logical sequence of activities. It may be calculated after the activities have been scheduled. It determines the earliest expected project finish date.
\end{defn}

The \nameref{cp} is essentially a network of tasks. They are ordered in time and can have a parallel structure. the path with the most time is the \nameref{cp} and it has no float. Delays in the \nameref{cp} will cause the project to delay. This method of time management helps us decide where to allocate scare resources.\\

\subsubsection{Path Techniques}

There are various techniques we can use to optimize the critical path. If we can use more resources, like staff or equipment, we can start doing jobs in parallel. If we are able to take on more risk, we can fast track and overlap sequential activities. The main thing we want to do in every case is to maximize compression for least cost.


% Stanley Yen
\section{Constituents of Nucleons}

\subsection{Quantum Review}

There are a few things we must understand about quantum physics. Light is a wave because it shows interferences and it is also a particle by the photoelectric effect.

\begin{defn}[Photoelectric Effect]\label{pe}
Light energy comes in packets of size. Where $h$ is Planck's Constant and $f$ is the frequency of light.
\[ E =hf \]
\end{defn}

This is a general property of all matter, not just of light. It is possible to see diffraction patterns made by beams of electrons, neutrons, protons, carbon atoms, and even Buckyballs!

This is the fundamental principle of wave-particle duality.

\begin{defn}[De Broglie's Wavelength]
  The wavelength of particle is given by $\lambda = h\cdot p$ where $p$ is the particle's momentum
\end{defn}

By \nameref{hup}, the act of measurement disturbs the system being measured.

\begin{defn}[Uncertainty Principle]\label{hup}
The uncertainty principle is any of a variety of mathematical inequalities asserting a fundamental limit to the precision with which certain pairs of physical properties of a particle known as complementary variables.
\[ \sigma_x \sigma_y \leq \hbar/2 \]
\end{defn}

In this lectures we shall apply these principles of quantum mechanics to several topics of subatomic physics.

\begin{itemize}
  \item The size and shapes of nuclei
  \item The energy scale of nuclear processes
  \item The existence of quarks
\end{itemize}

\subsection{Sizes and Shapes}

If we want to see nuclei, we can't just put it under a microscope.The wavelengths of the light used is proportional to it's resolution and visible light is pretty bad in this case.

However, if that is the case, how can we possibly see nuclei? The answer is scattering, however, the analogy of bullets through a haystack is not accurate due to the the wave effects of particles on this scale. As mentioned previously, electrons are not only particles but waves as well, when we intercept a beam with nuclei, we will get diffraction pattern of electrons that we can then study.

\subsubsection{Experiment}

When a group of waves hit an object with some sort of internal structure, the points act like a point source of waves. If they are far apart, there will be many peaks. If they are closer together we have a maximum, and if they are a single particle, we get a single wave crest. The nucleus however is not an infinity small point. It is really more like a charge smeared over a volume. However we have a method of using this approximation to get better results.

\[ S_{spaced} = S_{pointlike} \cdot |\int \rho(r') e^{iqr'}dV'| ^2 \]

For elastic scattering at angle $\theta q = 2p / \hbar * \sin(\theta/2)$

By measuring the scattering rate at different angles $\theta$, and sampling at different values of $q$, we can map out the square of the Fourier transform.Then apply an inverse Fourier transform to get the nuclear charge density $\rho(r')$

To do this, we have an electron beam collide into target some times the electrons interact with the nucleons and bounce off. We can use a magnetic spectrometer to look at the distribution of electrons post-interaction. By swinging this spectrometer at different angles, we sample the space. \\

\begin{center}
[scattering image]
\end{center}

\subsubsection{Interpretation -fix}

When we look at the actual distributions, the largest peak represents the elastic scattering while the rest of the bumps are always inelastic, there the electron happened to excite the nucleus (something inside the nucleus).\\
Since a small object gives a large diffraction and vise versa, we can calculate use the data to calculate the radius of nuclei. The narrower the diffraction the larger the radius of the nucleus.\\
If a atom was the size of football stadium, the nucleus would be on the scale of a pea. Yet it it 99.97\% of its mass

\subsection{Nuclear Energy Scale}

Similar to electrons, protons and neutrons have shell like structures. However, nuclear energies are very high. This can be explained by the \nameref{hup}. Since nuclei are 45000x smaller than atomic scales, by the \nameref{hup}, since $\Delta x$ is very small, $\Delta p$ is very high.

\begin{align*}
p_n / p_a &= 45000\\
KE_n / KE_a &= (45000)^2 / 1836 = 1.1\e{6}
\end{align*}

\subsection{Quarks}

Later on the question was posed to whether or not protons and neutrons were the last of the particles. The answer came in the form of using narrower and narrower wavelengths to probe the nucleons further.

What we can do is shoot electrons at a film of solid hydrogen and see how the protons diffract. Back in the day, 1/100 fm was the attainable wavelengths of those electrons, much smaller than the current accepted radius of a proton. This meant that we could definitely have resolution of things smaller than the protons if they were indeed as small we thought they were.\\

Looking at the diffraction pattern we noticed that it wasn't equal, it wasn't a single crest. Instead it looked similar to our studies of nuclei. The plots showed elastic collisions happening with large amplitudes and a few bumps, signalling inelastic scattering. This pattern meant that there were indeed internal `things' that could be excited. this meant that nucleons could be excited. Moreover, we notice that at very large angles, evidence suggests that there are point like objects inside the proton as the bumps smooth out to supposed point like particles.\\

Three quarks for muster mark.

\section{Nuclear Structures}
This talk summarizes our knowledge of nuclei and nuclear structure. First there was a brief over of the common units and unit scales and along with definitions if isotopes, isobars, isotones, types of decay and the chart of nuclides.

\subsection{Binding Energies}

\begin{defn}[Binding energy]\label{be}
Binding energy is the mechanical energy required to disassemble a whole entity into separate parts. A bound system typically has a lower potential energy than the sum of its constituent parts, this is what keeps the system together.$^{[W]}$
\end{defn}

Recall $e=mc^2$, the main idea here is that the relationship between mass and energy are shared. However, the proportion of scaling is negligible at the macroscopic level. However, at the subatomic level when we bind two particles together, they decrease their energy and therefore decrease their mass.

\begin{exmp}[Atomic Mass from Binding]
\[  m_{atom} = m_{nucleus} + m_{electrons} - \frac{b_{e}}{c^2} \]
\end{exmp}

Nuclear binding energies however are much larger than atomic binding. In most cases, the atomic binding energy is usually just a tiny correction term that may be omitted. In fact, we look at most tabular data you will find that it will be in an atomic energy which we can convert to nuclear by using the correctional terms.\\

\subsection{Saturation}
Consider the term $B/A$, it is the average binding energy per nucleon. Empirically, the average $B/A$ is $8 MeV$ (Average of an Average? Yes please!). The plot itself peaks at iron and slowly decreases as the increasing repulsion of the protons causes us to pack more and more neutrons in to introduce enough strong force to balance the electrostatic. This is called the saturation of nuclear forces.

\begin{center}\label{plot}
 \includegraphics[scale=0.5 ]{../img/ba}
\end{center}

\begin{defn}[Saturation of Nuclear forces]\label{sat}
The nucleus has become large enough that nuclear forces no longer completely extend efficiently across its width. Attractive nuclear forces in this region, as atomic mass increases, are nearly balanced by repellent electromagnetic forces between protons$^{[W]}$
\end{defn}

At the size individual nucleons only feel a neighborhood of forces. This indicates that the nuclear binding forces must be short ranged, similar to the forces of a kettle of water; it is the short ranged van der waals forces that hold it together.
\\
\[ e_{boil} = e_{binding} \propto A^2_{water} \]

This is constant due to the short range of the forces.

\[ B \propto A^2 \implies B/A  \propto A\]

This $8 MeV$ is about $1\%$ the total mass of the proton. This means that the mass of a nucleus is nearly $1\%$ smaller than the mass of it's constituent nucleons. Now its not on the order of $10^{-9}$ it is not negligitable.

\subsection{Studing the plot (\ref{plot})}

By looking at the plot from \ref{plot} plot we can learn a lot about nuclear processes.
The first peak is related to the production of $He^{4}$ in the sun, which release 28MeV per fusion! That part of the plot is where most stars spend their lives. Successive steps take far less energy. This helps us understand other properties of stars by looking at the Hertzsprung-Russell diagram.

\begin{center}\label{rus}
 \includegraphics[scale=0.23]{../img/rus}
\end{center}

\begin{itemize}
  \item H 7 myr \
  \item He 500 kyr \
  \item C 600 yr \
  \item Ne 1 yr \
  \item Si 1 d \
  \item Core Collapse < 1s
\end{itemize}

\subsubsection*{Going Backwards}

Also notice that if you go backwards along the plot, if we split up the large nucleons they will release energy as they go back up the potential. This is how we understand and make use of the energy in reactors. splitting $U$ for example can give us over $200+ MeV$.

\subsection{TRIUMF}

To figure out these binding energies, we can use something like TITAN and a beam of unstable nuclei. TITAN uses what we call a Penning Trap. It is a strong homogenous magnetic field and weak electrostatic field. By using the following equation we can accurately measure the mass of the particle inside the trap.

\[ v_c = \frac{1}{2\pi}\frac{q}{m}B\]

The type of error we get is on the order of $10^{-12}$ enough to detect the $1\%$ changes in mass. TITAN is so accurate that if we were to weigh a plane, the error would be the mass of a missing screw.

\subsection{Liquid Drop Model}

The liquid drop model in nuclear physics treats the nucleus as a drop of incompressible nuclear fluid. It was first proposed by George Gamow and then developed by Niels Bohr and John Archibald Wheeler. The fluid is made of nucleons (protons and neutrons), which are held together by the strong nuclear force. This is a crude model that does not explain all the properties of the nucleus, but does explain the spherical shape of most nuclei. It also helps to predict the binding energy of the nucleus.

\begin{defn}[Semi-empirical mass formula]\label{ldm}
\[ E_b = a_V A-a_S A^{2/3} - a_C \frac{Z^2}{A^{1/3}} - a_A \frac{(A-2Z)^2}{A} - \delta(A,Z) \]
\end{defn}

\begin{center}\label{rus}
 \includegraphics[scale=0.23]{../img/liquid}
\end{center}

\subsection{The Valley}

If we look at the isobars of a certain mass we can actually notice that mass vs z ha a very parabolic fit where the vertex also happens to be the most stable point. In fact, if you have a nuclide on the side of the valley, it will roll down the walls of the potential well by $p \rightarrow n$ or vise versa undergoing $\beta^{\pm}$ decay. At the tops of the valley are what we call drip lines. At the edges, the binding energy has gone to zero meaning that it is impossible to add more protons or neutrons because they just drip away!

\section{Nuclear forces and mesons}

On the scale of atoms, the quantum effects are quite large. What we have to solve these kinds of problems is the schrodinger wave equation.

\begin{defn}[Wave Equation]\label{swe}
\[ -\frac{\hbar^2}{2m}\bigtriangledown^2\psi(r) + V(r)\psi(r) = E\psi(r) \]
\end{defn}

\subsection{Coulomb Force}

Imagine a hydrogen atom where the coulomb potential of the proton was $V(r) = -ke^2/r$. When you solve the \nameref{swe} we can use what Planck discovered, $E=hf$ and get quantized spectral lines. This tells us that the electron and proton relationship is in fact a coulomb potential.

At that time there was no evidence for any other forces effected the atom. However there clearly would have be a force within the nuclei that attract the neutrons and protons that were very short range. Otherwise the protons would repel.

\subsubsection{Rutherford's Data}
One way we tried to study it was to probe the nuclei with high energy protons. The scattering data as mentioned before is a way of sampling the fourier domain of the deflection from electrostatic repulsion. This was the equation predicted for the deflection with only the coulomb force..

  \[ p(\theta) = eQ^2 / e^2\sin^4(\theta/2) \]

However with Rutherford's data, we were not able to find any difference between the evidence and the proposed results. Later on, with more energetic experiments, they could get the protons close enough to the nuclei that the short range force perturbed the predicted scattering.

\subsection{Comparing Long and Short Range Forces}

Understanding the nature of this short distance force can be best explained with comparing a long range force like gravity with a short range force like the van de waals.

Consider the binding energy of a planet. The larger the planet the higher the escape velocity per unit mass. Now consider the energy it requires to vaporize a body of water. A cup, a bathtub, and ocean? It does not get any harder to boil the water per unit mass.

The same thing happens with the strong force. After say, $C^{12}$ the range of the force can only feel so many particles. That way after $C^{12}$ it becomes hard to really feel more particles. This is what \nameref{sat} is really about. In fact, we can actually use $C^{12}$ as a tough estimate for the range of the strong force, about $5fm$. It is actually more like $2-3 fm$

\subsection{Probing the Nuclei}

There are two techniques we use to study these kinds of properties. One is called scattering and the other is called bound state.

\subsubsection*{Scattering}

This is essentially what Rutherford did back in the day. We throw particles into a potential and see how the potential changes it's trajectory. By sampling $V(r)$ and finding it's inverse fourier transform, we can figure out the shape and amplitude of the potential.

\begin{defn}[Born Approximation Scattering Amplitude]
The scattering amplitude is the Fourier transform of the Scattering potential $V(r)$
\begin{align*}
  A = \int \psi^{*}V(r)r\psi dr = \int e^{iqr}v(r)dr = fft(v(r))\
\end{align*}
\end{defn}

\subsubsection{Bound State Spectroscopy}
Just like atoms and molecules, nuclei exhibit a rich and complicated spectra of excited states, and these can tell us about the nuclear forces holding the nucleus together.

\begin{align*}
  ^12C(\alpha,\gamma)^16O\\
\end{align*}
Tigress gamma ray spectrometer now under construction in ISAC-II experimental hall. Tigress is position-sensitive to allow precise compensation for Doppler shift due to motion of the recoiling nucleus.

\begin{defn}[Parity]
Tells whether the wavefunction is even or odd when $x \rightarrow -x$, wave functions are either even or odd functions.
\begin{align*}
  \Psi(-x) = \Psi(x) &\implies \pi = +1 \text{ even}\\
  \Psi(-x) = -\Psi(x) &\implies \pi = -1 \text{ odd}
\end{align*}
\end{defn}

\subsubsection{Important Features of the Nuclear Binding Force}

\begin{enumerate}
  \item Short-ranges (a few fm)
  \item Attractive at the distance > 0.6fm - that's what binds the nucleons together in a nucleus
  \item Strongly repulsive at short distances of < 0.5 fm - that's why nuclear matter is highly incompressible, and this causes the outward bounce. of the shock wave in a core - collapse supernova.
  \item \nameref{ssd} - quite unlike electromagnetic interactions in an atom or molecule.
  \item Doesn't distinguish between p - p, p - n or n - n, as long as they are in the same spin orientation.
\end{enumerate}

\subsubsection{Strong Spin Dependence}\label{ssd}

Recall that protons and neutrons are spin $1/2$ particles. They have intrinsic angular momentum $1/2$ in units of $h / 2 \pi$. Relative to some direction z.

Now consider the Deuteron. ($^2H$) It has a spin $J=1$ and consists of a proton and neutron with parallel spins with relative orbital angular momentum of $L=0$.If we try bond a proton and neutron with anti-parallel spins, the system will not bind together. It instantly falls apart.\\
This is quite unlike the H atom, where the spin parallel and spin anti-parallel orientations result in a tiny splitting of the 1s level -- the origin of the 21 cm radio emission that radio astronomers use to map out hydrogen in the galaxy.\\
Nuclear forces don't distinguish between protons and neutrons (neglecting the Coulomb interaction) as long as the two nucleons involved are in the same spin orientation. If the nuclear forces don't distinguish between protons and neutrons, then in some sense, we can regard protons and neutrons as two manifestations of the same particle.

\begin{defn}[Nucleons and Isospin]\label{isospin}
  By analogy, nucleons are isospin-$1/2$ particles ($l=1/2$) with two possible isospin states, $l_3=+1/2$ for the proton and $l_3 = -1/2$ for the neutron.
\end{defn}

\subsection{Mesons}

Later on the question became, 'what causes the nuclear forces?'. Yukawa made the analogy of fields in electromagnetism, charged particles exchange photos to interact, since photons are massless, they have $\infty$ range.

In 1935, Yukawa proposed the meson, the photon equivilant of the strong force, Suppose that mesons have mass $m$, if a nucleon violates energy conservation to create a meson, we can use \nameref{hup} to calculate the time it can last.

\subsubsection{Yukawa Hypothesis}

In the early 1930's it was know that the nuclear forces had a range of $\approx 2\text{fm}$

Analogy with electromagnetism : Two charges bodies do not interact by action at a distance but with an electric field interacting with another body. This field can be quantized as photons with no mass and infinite range.\\
In 1935 Yukawa postulated that the strong nuclear force is carried by quanta called mesons. Nucleons interacted by exchanging mesons.\\
Supposed that the mesons have a mass $m$. Creating a energy violation with \nameref{hup} will result in

\[ \Delta t \le \frac{\hbar}{\Delta E} = \frac{\hbar}{mc^2}\]

Since we know that a meson can travel a distance of $R = c\Delta t = \frac{\hbar c}{mc^2}$ using $\hbar c = 197.3 \text{MeV-fm}$ and knowing that $R \approx 2 \text{fm}$ we get that the mass should be around $100 \text{MeV}$. With the mass between electrons and nucleons there were called middle weight mesons.

\begin{defn}[Yukawa Potential]\label{yp}
  The exchange of mesons of mass $m$ gives rise to a potential.
  \[ \Phi = ge^{\alpha r}/r \alpha = \frac{mc}{\hbar} \]
\end{defn}

In the limit $m\rightarrow0$ we get back the familiar coulomb potential.

\[ \Phi = g \frac{1}{r} \equiv \frac{e}{4\pi\epsilon_0}\frac{1}{r}\]

So we can think of $g$ as the "strong charge" giving rise to a meson potential $\frac{ge^{-\alpha r}}{r}$ in the same way that $e$ is the electric "electric charge" giving rise to an electric potential $\frac{e}{4\pi\epsilon_0}\frac{1}{r}$ The range of the Yukawa potential is approximately $\frac{1}{\alpha} = \frac{\hbar}{mc} = \frac{\hbar c}{mc^2}$, exactly what we got with the \nameref{hup} argument.\\
In 1947 scientists went to the sky to look for cosmic rays for mesons using stacks of photographic. Usually, cosmic rays of protons around TeV collide with a nitrogen nucleus that break apart into pion -> muon + neutrino.

\subsubsection{From the Heisenberg uncertainty principle}

The shorter the lifetime $\Delta t$ of a quantum state (such as a meson) the greater the uncertainty in its mass or energy $\Delta E$.

\[ \Delta E \cdot \Delta t > h/2\pi\]

So when these short-lived mesons are produced in a high energy particle collision, and you try to measure their mass, they show a width due to the uncertainty principle.
The nucleus is a dynamic object where pions and heavier mesons constantly flit in and out of existence for only as long as permitted by \nameref{hup}. These temporary mesons that exist by energy borrowed from the uncertainty principle are called virtual mesons.

\section{Nuclear Shell Model}

Electrons in atoms take up shells with discreet energies. The question was that whether or not nuclei had the same behavior. This was in fact the case, different shapes in electron shells give us different behavior of the atom, be it radius or ionization. Most of it's behavioral patterns can be summarized in the table of elements.

The problem was that the success of the liquid drop model did not posit the potential discreet jumps. Furthermore, when we look at the binding energies, nucleons should not have well defined energies. Again, more fluid than discreet. Many papers were written about why they couldn't have shells. However later experiments proved them wrong.

\subsection{Magic Numbers}

The experimental data compared to the liquid drop model shows excess binding energy. If we look at the ionization analog for atoms. We notice jumps at nucleon numbers (8, 20, 28, 50, 82, 126). These numbers seemed to indicate extra binding. We look at even even nuclei, we see a another maximin at (50, 82, 126) we get a extra enhancement of binding. Moreover, we can look at the analog of reactivity (capture cross section, we find dips at (20, 50, 82, 126). When we plot nuclear radius, we also find minimums at (20, 28, 50,82, 126). So the data indicate that at (2, 8,20, 28, 50, 82, 126), these where magic numbers

By looking at the 1D infinity square well where,

\[E = p^2/2M = n^2h^2 / 8ML^2 \]

If we pick a different shaped well. If we use a 3D infinite square well or a harmonic well, it only works closer to the lower numbers. Even a Woods Saxon potential, one that fits nuclear potential well, does not fit the shell closures.

\subsection{Fermi}

A spin orbit force $v(r) = V_o(r) = V_{ls}\vec{L}\cdot\vec{S}$. Since the protons and neutrons have spin, the force is depending on the parallelism or antiparallelness of two particles. The parallel configuration is lower in energy and the antiparallel configuration is higher. This effect is very small in atomic physics.However, in nuclear physics, it is huge. If we use this spin force to predict the shell closures, it turns out that it is what accurately models the shells. They got the nobel prize!

\subsection{Filling the Shells}

How that we have a model of the nuclear shells, we can start filling them. It is important to have two different shells, one for protons and one for neutrons. How it is much easier to explain stable nuclei. The spin dependent shell model explains the stability of magic numbers and even even nuclei. It is also important to note that any unpaired nucleon will result in the spin of the whole nuclei. While complex for excited nuclei, it is a very simple model for ground state nuclei.

\subsection{Alpha Decay}

$\alpha$ particles are actually $^4$He nuclei. It is what heavy nuclei do to reduce the ratio of $Z^2/A$ by spontaneous emission. Notice that $Z^2/A$ is the ratio between repulsive coulomb potential and strong attractive potential.

However, it seems strange that it would emit protons vs alpha. However, because of the tight mass of the alpha particle, its stability, there is more energy available to actually emit the particle. If we count the number of quantum states.

\begin{defn}[Geiger Nuttall Law]
   In nuclear physics, the Geiger - Nuttall law or Geiger - Nuttall rule relates the decay constant of a radioactive isotope with the energy of the alpha particles emitted. Roughly speaking, it states that short-lived isotopes emit more energetic alpha particles than long - lived ones
   \[\ln \lambda = -a_1\frac{Z}{\sqrt{E}}+a_2\]
\end{defn}

This was very strange and defied a lot of physics that time. however it was later discovered that it was a result of quantum tunneling, an analog of evanescent waves in optics.

\begin{defn}[Evanescent Wave]
  An evanescent wave is a near-field wave with an intensity that exhibits exponential decay without absorption as a function of the distance from the boundary at which the wave was formed
\end{defn}

\begin{center}
 \includegraphics[scale=0.7]{../img/qwave}\\
 We can essentially see that the wave decays exponentially in the potential wall.
\end{center}

\section{TITAN}

Measuring the mass of short-lived isotopes with high precision
Radioactive isotopes from ISAC are sent to TITAN to undergo cooling, charge-breeding and trapping. The entire process occurs in about 10 milliseconds, allowing radioactive isotopes with short half lives to be studied.

\subsection{Why mass?}

There are a range of physics and chemistry that require precise measurements. We even text QED and nuclear physics. Many of the time with nuclear structures and sciences we need precisions that go from $10^{-7}$ to $10^{-9}$. That mass really is for us is binding energy. It gives us a handle of how the sum of the physics goes on. Moreover, masses are unique and helps us identify the element. It also allows us to calculate what is possible in terms of nuclear reactions.

\begin{center}
P for positive, P for possible.
\end{center}

\subsection{Ion Trapping}

Ion trapping is an ideal laboratory with 3d confinement. We have dell defined fields because we understand Maxwell's equations. If we can, harmonics, acumination and easy ion manipulation. The two most common types of ion traps are the Penning trap and the Paul trap (quadrupole ion trap).

\begin{defn}[Earnshaw's Theorem]\label{earnshaw}
A collection of point charges cannot be maintained in a stable stationary equilibrium configuration solely by the electrostatic interaction of the charges. This was first proven by British mathematician Samuel Earnshaw in 1842. It is usually referenced to magnetic fields, but was first applied to electrostatic fields.
\end{defn}

With that said we can either have rotating fields or introducing an magnetic field called a penning trap.

\subsubsection{Cyclotron Frequency}

Mass measurement via determination of cyclotron frequency.

\[ v = \frac{1}{2\pi}\frac{q}{m}B\]

\subsection{Penning Trap}

A Penning trap is a device for the storage of charged particles using a homogeneous static magnetic field and a spatially inhomogeneous static electric field. This kind of trap is particularly well suited to precision measurements of properties of ions and stable subatomic particles which have a non-zero electric charge.

\begin{defn}[High Precision]
In order to increase the precision of the experiment modeled by the following,

\[ \frac{\sigma m}{m} \approx \frac{1}{v_cT_{RF}\sqrt{N}}\]

\begin{itemize}
  \item Increase charge
  \item Increase field
  \item Limited by Halflife
\end{itemize}
\end{defn}

\subsubsection{RFQ TRap}

A quadrupole ion trap or quadrupole ion storage trap (QUISTOR) exists in both linear and 3D (Paul Trap, QIT) varieties and refers to an ion trap that uses constant DC and radio frequency (RF) oscillating AC electric fields to trap ions.
With the He buffer gas, it will accumilate the cooled particles in a potential

\subsubsection{The Electron Beam Ion Trap}

The EBIT device uses the combination of electrostatic and magnetic fields to confine particles in 3-D. A magnetic field is applied so as to trap particles in 2-D, by virtue of their cyclotron motion, and then electrostatic electrodes are used in order to trap in the third dimension.

Once trapped an intense electron beam is fired upon the ions. This beam strips electrons from the ions and hence their charge state rises. A higher charge state is required to reduce statistical uncertainty in the mass measurements taken in the penning trap.

\subsection{Island of Inversion}

An island of inversion is a region of the chart of nuclides that contains isotopes with a non-standard ordering of single particle levels in the nuclear shell model. Such an area was first described in 1975 by French physicists carrying out spectroscopic mass measurements of exotic isotopes of lithium and sodium. Since then further studies have shown that neutron-rich isotopes of five elements, 11Li, 31Na, 36Mg, 38Si, and 48Ca belong to one such region.

% Jens Lassen
\section{Isotope Ion Sources and Beams}
the linear accelerator is required to overcome the coulomb repulsion
foils are usually tantalum or niobium. these foils are also dimpled so they don't stay together. The targets are
40mm long and is heated to white glowing tempuratures in order to generate ions. By wrapping the tube with layers of tantalum, we can insulate it more up to 2300$^\circ C$.
the thickness of the foils are dependent on the life time of the material. (radioactive material that is). design:
one thine we nmay consider is that although we may be able go generate larger targets, they will produce more
fission per second but will not be as efficient. the problem now is that it is hard to take the isotopes to the ion source. photo fission: if we have a gemma ray at the right energy we can make the uranium nuclei vibrate in the way we want and get it to break apart.
ARIEL: plans to double science output with one proton and one electron. afterwards we plan to add another proton beam that will result in tripled beam production after the complete of the secondary proton beam line.


%John Behr
\section{Atom Traps and Wrong-Handed Neutrinos}

questions to be answered : how are magnetic optical atoms traps work? what can be trapped? How can we simulate parity violation experimentally.
atom tests in francium the heaviest alkali atom.

\subsection{Magneto-optical trap: damping}

we want a damped oscillator. the damping part is easy. we have beams that are lower frequency than atomic resonance. what happens is that the atoms vibrate towards certain beams so that the incoming light redshifts and gives the correct energy level. that ways random vibrations get dampened.

\subsubsection{Why optical traps dont work}

again as mentioned before it is \nameref{earnshaw} that prevents regular optical trap. Instead with can use time dependent forces, dipole force traps or modify internal structure.

the following equation is Poynting's theorem:

\[ \bigtriangledown \cdot \vec{S} = \frac{c}{4\pi}\bigtriangledown \cdot (\vec{E} \times \vec{B}) = -J\cdot E - \f{\partial u}{\partial t} = 0 \]
% note: by overshooting the driving force frequency, you flip the oscillation 180$^\circ$

\subsection{Neutral Atom Trap}

we only trap about 1/1000 atoms from background. then we transfer about 75\% of the material into another trap that reduces noise and actually has detectors.

\section{Ramo's Theorem}

Ionizing radiation integrating with matter liberates positive and negative charge carriers. This usually leads to an electron and an ion. Our typical detector just has a voltage thought them to collect the ions and electrons. However you will get a time vary current on the anodes when the charges drift. this means that our signal is spread over a period of time.

\[  \bigtriangledown \cdot E = \frac{\rho}{e} \]
\[  \bigtriangledown \cdot V = -E \]

Quick notation $ ^XV_j $ where V is the important quantity, j is the part of the detector and x is the thing we are interested it.

As far as we know, calculating charge trajectories is easy. At lower fields, they move at a velocity proportional to the electric field. Do note that it does break down as field gets stronger. Its very similar to terminal velocity.

\begin{defn}[Superposition]
  When you put charges together the electric fields are a sum of individual charge.
\end{defn}

If you were to take a volume integral of $ ^xV(r)\bigtriangledown^{2} {}^{x} V(r)$ becomes:

\[ \iiint_a(^xV(r){\bigtriangledown^{2}} ^xV(r))dr^3 = {\frac{-1}{e}} {}^xV_a ^xQ_a \]

Yo, point is. This equation holds. It is purely mathematical but it works.

\[ \sum {}^F Q_j {}^G V_j = \sum {}^G Q_j {}^FV_j\]

\section{Antihydrogen} % (fold)
\label{sec:antihydrogen}

the antihydrogen produced at ATHENA are nearly at rest but still drift towards the wall. These particles were not trapped and required a lot of drift room. ALPHA was designed to confine these particles and do spectroscopy. The alpha collaboration is about 40-5o people from various countries.

\subsection{Motivations} % (fold)
\label{sub:motivations}

There are many motivations to study antimatter.  While we know a bit about how matter works antihydrogen is the simplest one to work with.
The big annihilation in early universe.1:10 000 000 000 photon to matter.

The problem is called haryogenesis. How do we get matter out of a symmetric universe. So we study antimatter we might find small differences.
% subsection motivations (end)

\subsection{Symmetry} % (fold)
\label{sub:symmetry}

when we discovered that the weak force violated a lot of symmetries. CPT assumes that Quantum theory, lorenz invariance of special relativity and statistics. This suggests that antimatter has to be the same as hydrogen. If we find that the spectra or mass is different, we will have to rewrite the books.

% subsection symmetry (end)

\subsection{Basics} % (fold)
\label{sub:basics}

\section{Nuclear Reactions}

Lets determine the energy needed to roe a 10fm nucleus. Once we do the math we find that the energy required is about 10MeV which is huge for something like a photon. Instead we use nucleons that allows us to probe the nuclei.

\[ T + b \rightarrow R + e\]

where they stand for Target, projectile, recoil and ejection.

\subsection{Types of Reactions}

Elastic scattering means that internal states and identities of nuclei are unchanged.
Opposed to inelastic where particles (most commonly the recoil) become excited.
Transfer reactions happen when bot the projectile and target are transmuted by a transfer of one or more nucleons.
\begin{itemize}
    \item Pick up
    \item stripping
\end{itemize}
Fusion evaporation reactions cause the projectile and target fuse and excites it. The compound nucleus then deexcites by evaporating nulceons and/or $\alpha$ particles.
Spallation reactions: when the projectile breaks the target up into a relatively large unmber of reaction products.
\[181Ta + p \rightarrow 11Li + \alpha + p + 166ER\]

Fission: when heavy target nucleus breaks up int roughly equal mass fragments.

\subsection{Reaction Probability} % (fold)
\label{sub:reaction_rop}

The classical interpretation of a projection in terms of colliding spheres. Coulomb scattering is the EM force causing the scattering between charged particles even in absence of the nuclear forcProbing atoms are hard. Laster spectroscopy is probing the nuclear structure using high precision measurements of the affect of the nuclear charge and magnetism distributions on the atomic electrons.Probing atoms are hard. Laster spectroscopy is probing the nuclear structure using high precision measurements of the affect of the nuclear charge and magnetism distributions on the atomic electrons.  e. Reactions are als not very selective. Not all reactions populate the same energy levels or states. The yield to different states reflect the underlying structure as will as spin and parity selection rules. Resonance also happens to effect the cross sections at certain energy levels that are affected by parameters like the characteristics of the nuclei. At low energies they are very distinct but as we get larger energies they resonance spectrum become featureless as these resonances overlap and mix.

\subsection{Astrophysics} % (fold)
\label{sub:astrophysics}

We observe observe the abundance of elemental abundances by looking either by start light, or pulverizing meteorites.

\section{Measuring ground state nuclear properties at TRIUMF}

Probing atoms are hard. Laster spectroscopy is probing the nuclear structure using high precision measurements of the affect of the nuclear charge and magnetism distributions on the atomic electrons.

\[ \Delta E_{hfs} = A\frac{K}{2} + B\frac{\frac{3}{2}k(k+1) - 2l(l+1)J(J+1)}{4l(2l-1)J(2J-1)}  \]

if the same atomic transition is viewed in different isotopes of the same elements. you will see that the center of mass energy is different along with the volume. This causes the energy spectra to be different.



\end{document}
